\chapter{Introduction to Probability}
 \label{chap:probability} \index{Probability}
 \index{Kernels}

 \labsection{Simple Probability}
For most people, the idea of simple probability is pretty intuitive.
The probability of rolling a 5 on a 6-sided die is ${ 1\over 6}$ and
the probability of drawing the queen of hearts in a 52-card deck is
${1\over 52}$.  Generalizing this idea yields the equation:

\begin{equation}\label{eq:simpleProb}
  P(\text{event~ A}) = {\text{No.~of~ways~A~can~happen}\over \text{Total~no.~of~things~that~could~happen}}
\end{equation}


\labsection{Probability of multiple events}

Things get a little harder when you consider multiple events because
you have to consider whether the outcome of one event will affect the
outcome of the next.  For example, suppose you roll a $6$-sided
die twice and you want to know the probability that both rolls come up $5$.
Having a $5$ come up on the first roll doesn't affect the probability
that a $5$ comes up on the second.  These two events are independent
of one another.  The outcome of one event does not affect the outcome
of the other.

In contrast, consider randomly selecting marbles out of a bag which initially
contains $3$ red marbles and $8$ green marbles.   If you reach into the
bag and grab two marbles, what is the probability that they will both
be red?  Notice that selecting a red marble on my first reach into the bag
modifies the number or red marbles available for selection on my
second reach into the bag. Hence the two probabilities are dependent
on each other.  This is an example of dependent events.

\subsection*{Intersection of two events (event A \textbf{and} event B)}
\marginfig[-1in]{Figures/twoDieRollsII}{\label{twoDieRolls}All 36
  outcomes from rolling two fair dice.  Out of 36 possible outcomes,
  only one yields a $5$ on both rolls.  Hence the probability of
  rolling two $5$'s is ${1\over 36}$.}

We'll start with the case where two events, $A$ and $B$, both
occur. (Also called the intersection of two events. See figure \ref{fig:VennDiagram})
The math for this sort of situation is:

\begin{equation}
  P(A \cap B) = P(A)  \times P(B)\nonumber
\end{equation}

but let's see some examples so it makes more sense.


\subsubsection*{Independent Events}
Consider rolling a fair $6$-sided dice.  What is the probability that
$5$ comes up on the first two rolls?  Your intuition is probably telling
you that rolling two $5$'s is less likely than rolling just one, so
you expect the probability to be less than ${1\over 6}$.  In figure
\ref{twoDieRolls} you will see all 36 possible outcomes when rolling a
fair dice two times.  Of those 36 possible outcomes how many of
occurrences of two $5$'s being rolled were there?  Just one.  Hence,
the probability is ${1\over 36}$.  You may have noticed that
${1\over 36} = {1\over 6} \times {1\over 6}$.  In other words, the
probability of event A happening \textbf{and} event B happening is
just the product of the two individual probabilities:

\begin{equation}\label{eq:andprobability}
  P(A\text{ and } B) = P(A) \times P(B)
\end{equation}



sometimes $P(A\text{ and } B)$ is written using the symbol for
intersection ($\cap$):
\begin{equation}\label{eq:andprobability}
  P(A \cap B) = P(A) \times P(B)
\end{equation}




\subsubsection{Dependent Events}
When the outcome of one event is affected by the outcome of a previous
event, we say they events are dependent.  For example, consider a bag
containing $3$ red marbles and $8$ green marbles.  If I reach into the
bag and select two marbles, what is the probability that they are both
red.  The probability that my first selection is red is:

\begin{equation}
P(A) = {3\over 11}  \nonumber
\end{equation}

\marginfig[-1in]{Figures/VennDiagram}{\label{fig:VennDiagram}If the
  blue circle represents the probability of A occuring and the red
  circle represents the probability of B occuring, then the overlap of
these two circles is the probability of both events occuring. By
sliding one cirle left or right we can adjust the size of the overlap
(aka $P(A|B)$)}

because $3$ out of theh $11$ choices are red.  What about the second
selection.  After the first selection, I am left with $10$ marbles,
$2$ of them being red.  Hence, the probability that I draw a red
marble on my second selection is:

\begin{equation}
 P(B) = {2\over 10}\nonumber
\end{equation}

The probability that event A (first selection is a red marble)
\textbf{and} event B is

\begin{align}
  P(A \cap B) &= P(A) \times P(B)\nonumber\\
              &= {3\over 11} {2\over 10}\nonumber\\
              &= {6 \over 110}\nonumber\\
              &= 0.055 \nonumber\\
              &= 5.5\%\nonumber
\end{align}



In general, the probability of two events occuring is given by:

\begin{equation}\label{eq:andprobabilityII}
  P(A \cap B) = P(A) \times P(B|A)
\end{equation}

where $P(A|B)$ is called ``conditional probability'' and means the
probability of $B$ happening \textbf{given that} $A$ already happened.
It's important to recognize that equation \eqref{eq:andprobability} is
the most general equation for two events and that equation  This equation reduces to equation
\eqref{eq:andprobability} when the events are independent.

We can do a little algebra on equation \eqref{eq:andprobabilityII} to
get:

\begin{equation}
  P(B|A) = {P(A \cap B)\over P(A)}
\end{equation}

Furthermore, if $A$ and $B$ are independent events, then $P(A\cap B) =
P(B \cap A) = P(B) \times P(A|B)$.  Hence, the above equation becomes:

\begin{align}
  P(B|A) &= {P(A \cap B)\over P(A)}\\
         &= \boxed{{P(A|B) P(B) \over P(A)}}
\end{align}


\marginfig[-1in]{Figures/twoDieRollsIII}{\label{twoDieRollsIII}All 36
  outcomes from rolling two fair dice.  <first roll> - <second roll>}

This is known as Bayes' Theorem or Bayes' Rule and is of profound
importance going forward.



\subsection*{Union of two events (event A \textbf{or} event B)}
